\chapter{Results} % Chapter title

\label{chapter:results} % For referencing the chapter elsewhere, use \ref{chapter:computational_neuro} 

%----------------------------------------------------------------------------------------

% Define some commands to keep the formatting separated from the content 
\newcommand{\keyword}[1]{\textbf{#1}}
\newcommand{\tabhead}[1]{\textbf{#1}}
\newcommand{\code}[1]{\texttt{#1}}
\newcommand{\file}[1]{\texttt{\bfseries#1}}
\newcommand{\option}[1]{\texttt{\itshape#1}}

%----------------------------------------------------------------------------------------

\section{Section}
\todo{Zusammenfassung der Resul-tate
Hier geben Sie wieder, was aus der Arbeit als Ergebnis resultiert. Es ist darauf zu achten, dass keine Bewertung der Daten vorweggenommen wird. Diese soll im Dis-kussionsteil erfolgen. Trotzdem sind die Daten und Resultate mit genügend Text zu erklären. Absolut zentral ist dabei eine präzise, treffende sprachliche Ausdrucks-weise. Von Alltagsslang und vagen Aus-drücken ist unbedingt abzusehen.
Bei grossen Datenmengen müssen die Rohdaten nicht zwingend publiziert wer-den.}

\newpage
