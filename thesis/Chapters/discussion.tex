\chapter{Discussion} % Chapter title

\label{chapter:discussion} % For referencing the chapter elsewhere, use \ref{chapter:computational_neuro} 


%----------------------------------------------------------------------------------------

\section{Multiple Query Optimization}

\todo{Bespricht die erzielten Ergeb-nisse bezüglich ihrer Erwart-barkeit, Aussagekraft und Re-levanz
Die Diskussion soll von einem differen-zierten, sprachlich präzisen Gegenüber-stellen von Fakten, Resultaten und Theo-rien geprägt sein. Persönliche Meinungen haben hier nichts zu suchen! Aussagen müssen durch (mathematische) Logik, wissenschaftliche Theorie oder Statistik begründbar sein. Wenn Vermutungen nicht begründbar sind, so sind diese nur dann festzuhalten, wenn ein Weg zu de-ren Begründung aufgezeigt werden kann, oder wenigstens eine wissenschaftlich plausible Erklärung existiert.
Interpretation und Validierung der Resultate
Rückblick auf Aufgabenstel-lung, erreicht bzw. nicht er-reicht
Nehmen Sie hier Bezug auf den Abschnitt 1.2!
Legt dar, wie an die Resultate (konkret vom Industriepartner oder weiteren Forschungsar-beiten; allgemein) angeschlos-sen werden kann; legt dar, welche Chancen die Resultate bieten.
Das weitere Vorgehen ist ebenso wichtig wie Ihre Arbeit. Jede wissenschaftliche Arbeit enthält offene Fragen oder Arbeits-schritte, die aus bestimmten Gründen nicht ausgeführt werden konnten. Diese sind aufzulisten und zu begründen.}
\newpage

\section{Quantum Neural Network}
