\chapter{Theoretical Fundamentals} % Chapter title

\label{chapter:theoretical_fundamentals} % For referencing the chapter elsewhere, use \ref{chapter:computational_neuro} 

%----------------------------------------------------------------------------------------

\todo{In der Regel ist zumindest ein kurzes The-oriekapitel notwendig. Es nimmt Bezug auf das thematische Oberthema, aber na-türlich nicht auf allgemeine theoretische Grundlagen etwa aus der Naturwissen-schaft.}
\section{Multiple Query Optimization}

A query\cite{codd_relational_1970} is a demand for information to be pulled from a database. A query can vary in size and structure, as well as in execution time. These are written with \code{SQL}, which is often specialized to the corresponding hardware\cite{shirgoldbird_microsoft_nodate}\cite{the_postgresql_global_development_group_postgresql_2022} it runs on. 

    
\begin{figure}[!h]
    \centering
    \begin{minted}{sql}
        SELECT * FROM USERS u
        WHERE u.NAME = "Abraham"
    \end{minted}
    \caption{This example SQL code would show all data from users with the name "Abraham"}
    \label{fig:sql_query_example}
\end{figure}

The example query in figure \ref{fig:sql_query_example} would be parsed and then used to pull and display data that is saved in the database. An exemplary view of such data is shown in table \ref{table:sql_query_result_example}.

\begin{table}[!h]
    \centering
    \begin{tabular}{|c|c|c|}
        \hline
        Name    & Surname & Gender \\ \hline
        Abraham & Martin  & M      \\ \hline
        Abraham & Tart    & F      \\ \hline
    \end{tabular}
    \caption{Exemplary data returned after executing the SQL query from figure \ref{fig:sql_query_example}}
    \label{table:sql_query_result_example}
\end{table}

Before execution, the queries themselves are taken apart and multiple execution plans generated\cite{microsoft_execution_nodate}. In a large system that handles multiple requests per second, these plans can be used to combine certain sub-queries and save execution time\cite{roy_multi-query_2009}. This by itself has been proven to be an \emph{NP-Hard} problem\cite{}. Nevertheless, finding the fastest combination of all plans can be done classically with an exhaustive algorithm in $O(n^2)$\footnote{Assuming that the amount of queries and plans per query are equal, if not then the runtime is $O(P^Q)$, where $P$ is the amount of plans per query, and $Q$ the total amount of queries}. There exist classical proposals that are faster than $O(n^2)$\cite{}, that whilst not finding the fastest combination, deliver results that still offer acceptable speed up. Hybrid approach combinations that use QAOA (short for Quantum Approximate Optimization Algorithm)\cite{} have shown to be quasi-optimal solution finders with a runtime of $O(I \cdot (PQ)^2)$.\par

\subsection{Query Dissection}







\newpage
