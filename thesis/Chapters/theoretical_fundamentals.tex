\chapter{Theoretical Fundamentals} % Chapter title

\label{chapter:theoretical_fundamentals} % For referencing the chapter elsewhere, use \ref{chapter:computational_neuro} 

%----------------------------------------------------------------------------------------

\todo{In der Regel ist zumindest ein kurzes The-oriekapitel notwendig. Es nimmt Bezug auf das thematische Oberthema, aber na-türlich nicht auf allgemeine theoretische Grundlagen etwa aus der Naturwissen-schaft.}
\section{Multiple Query Optimization}

A Query is a demand for information to be pulled from a database. A query can vary in size and structure, as well as in execution time. These are written with \code{SQL}, which is often specialized to the corresponding hardware it runs on. 

    
\begin{figure}[!h]
    \centering
    \begin{minted}{sql}
        SELECT * FROM USERS u
        WHERE u.NAME = "Abraham"
    \end{minted}
    \caption{This example SQL code would show all data from users with the name "Abraham"}
    \label{fig:sql_query_example}
\end{figure}

\newpage
