\chapter{Introduction} % Chapter title

\label{chapter:introduction} % For referencing the chapter elsewhere, use \ref{chapter:computational_neuro} 

%----------------------------------------------------------------------------------------

% Define some commands to keep the formatting separated from the content 
\newcommand{\keyword}[1]{\textbf{#1}}
\newcommand{\tabhead}[1]{\textbf{#1}}
\newcommand{\code}[1]{\texttt{#1}}
\newcommand{\file}[1]{\texttt{\bfseries#1}}
\newcommand{\option}[1]{\texttt{\itshape#1}}
%----------------------------------------------------------------------------------------

\section{Status Quo}
\todo{Nennt bestehende Arbeiten/Li-teratur zum Thema
→Literaturrecherche
Stand der Technik: Bisherige Lösungen des Problems und deren Grenzen
«Stand der Technik» ist ein Fachbegriff, der den aktuellen Stand des Wissens im Thema meint. Sie beweisen damit, dass Sie das Fachgebiet kennen und das we-sentliche Vorwissen aufgearbeitet haben.}


\section{Goals}
\todo{Formuliert das Ziel der Arbeit
Achtung: Ziel und Aufgabe sind nicht zwingend dasselbe! Bitte sauber trennen.
Verweist auf die offizielle Auf-gabenstellung des/der Dozie-renden im Anhang}

\newpage
